\documentclass[12pt,a4paper]{article}

\date{\today}

\usepackage{amssymb}
\usepackage{amsmath}
\usepackage[a4paper, margin=0.5in]{geometry}
\usepackage[colorlinks]{hyperref}
\usepackage[hiresbb]{graphicx}
\usepackage[round]{natbib}   % omit 'round' option if you prefer square brackets
\usepackage{wrapfig}
\usepackage[T1]{fontenc}
\renewcommand{\baselinestretch}{1.2} 
\usepackage{afterpage}
\usepackage{wrapfig}
\usepackage{floatflt}    % for floating figures with wrapped text

\usepackage{lipsum}    % for placeholder text
\newcommand\blankpage{%
    \null
    \thispagestyle{empty}%
    \addtocounter{page}{-1}%
    \newpage}

\renewcommand{\figurename}{Figura}
\usepackage{caption}

\hypersetup{
    linkcolor=red,          % color of internal links (change box color with linkbordercolor)
    citecolor=blue,        % color of links to bibliography
    filecolor=magenta,      % color of file links
    urlcolor=blue           % color of external links
}

\begin{document}

%\maketitle
\thispagestyle{empty}

\newpage
\pagestyle{empty}

\begin{center}\section*{Midterm Exam (part 1) - Computational Physics I}\end{center}
\vspace{0.2cm}
{\bf NAME: \verb|_________________________________________________|  SCORE:} \\
{\bf Date:} Monday 13 October 2025 \hspace{0.25cm} {\bf Duration:} 45 minutes\\
  \hspace{0.25cm} {\bf Credits:} 10 points (5 questions) \hspace{0.25cm}  {\bf Type of evaluation:} MT\\

\vspace{-0.6cm}

\noindent{\bf Please provide concise answers to the following items:}\\
\vspace{-0.5cm}
\begin{enumerate}

\item {\bf (2 points) Input/Output and data formats}\\
a. Describe $2$ distinct methods suitable for handling tabulated data input and output in Python.\\
b. List and briefly describe $2$ types of scientific data formats used in physics.\vspace{5.7cm}


\item {\bf (2 points) Systems of linear equations}\\
a. Explain how the Gauss elimination method for solving systems of linear equations works.\\
b. List the main steps for solving such systems via symbolic algebra with SymPy in Python.\vspace{5.7cm}


\item {\bf (2 points) Systems of nonlinear equations}\\
a. Indicate 2 methods that we can use to solve systems of nonlinear equations in Python.\\
b. Briefly explain how each method works.\newpage

\item {\bf (2 points) Data processing}\\
You have a simulation dataset consisting of  50 HDF5 files, each containing time, gas density ($\rho$), and velocity components ($v_x, v_y, v_z$). Each file corresponds to a different time in the simulation.\\
a. Write a Python function prototype for reading a single HDF5 file, computing the mean kinetic energy, and returning the time and the mean kinetic energy as NumPy arrays.\\
b. Sketch the loop needed to process all 50 files using the above function, so that you obtain and plot the mean kinetic energy versus time.\vspace{9cm}

\item {\bf (2 points) Image processing}\vspace{0.1cm}

\begin{minipage}{0.35\textwidth}
    \centering
    \hspace{-1.3cm}\includegraphics[width=5cm]{detail_S72-55208_orig.jpeg}
\end{minipage}%
\hspace{0.02\textwidth}
\begin{minipage}{0.45\textwidth}
Imagine you obtain this photograph of iron crystals from a scanning electron microscope (credits: NASA/JSC), and you are asked to isolate the more prominent crystals from the background and from the rest of the image. Design and sketch a suitable algorithm workflow to achieve this goal in Python.
\end{minipage}

\end{enumerate}

\end{document}
